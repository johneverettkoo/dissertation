% Options for packages loaded elsewhere
\PassOptionsToPackage{unicode}{hyperref}
\PassOptionsToPackage{hyphens}{url}
\PassOptionsToPackage{dvipsnames,svgnames*,x11names*}{xcolor}
%
\documentclass[
  11pt,
]{article}
\usepackage{amsmath,amssymb}
\usepackage{lmodern}
\usepackage{ifxetex,ifluatex}
\ifnum 0\ifxetex 1\fi\ifluatex 1\fi=0 % if pdftex
  \usepackage[T1]{fontenc}
  \usepackage[utf8]{inputenc}
  \usepackage{textcomp} % provide euro and other symbols
\else % if luatex or xetex
  \usepackage{unicode-math}
  \defaultfontfeatures{Scale=MatchLowercase}
  \defaultfontfeatures[\rmfamily]{Ligatures=TeX,Scale=1}
\fi
% Use upquote if available, for straight quotes in verbatim environments
\IfFileExists{upquote.sty}{\usepackage{upquote}}{}
\IfFileExists{microtype.sty}{% use microtype if available
  \usepackage[]{microtype}
  \UseMicrotypeSet[protrusion]{basicmath} % disable protrusion for tt fonts
}{}
\makeatletter
\@ifundefined{KOMAClassName}{% if non-KOMA class
  \IfFileExists{parskip.sty}{%
    \usepackage{parskip}
  }{% else
    \setlength{\parindent}{0pt}
    \setlength{\parskip}{6pt plus 2pt minus 1pt}}
}{% if KOMA class
  \KOMAoptions{parskip=half}}
\makeatother
\usepackage{xcolor}
\IfFileExists{xurl.sty}{\usepackage{xurl}}{} % add URL line breaks if available
\IfFileExists{bookmark.sty}{\usepackage{bookmark}}{\usepackage{hyperref}}
\hypersetup{
  colorlinks=true,
  linkcolor=Maroon,
  filecolor=Maroon,
  citecolor=Blue,
  urlcolor=blue,
  pdfcreator={LaTeX via pandoc}}
\urlstyle{same} % disable monospaced font for URLs
\usepackage[margin=1in]{geometry}
\usepackage{graphicx}
\makeatletter
\def\maxwidth{\ifdim\Gin@nat@width>\linewidth\linewidth\else\Gin@nat@width\fi}
\def\maxheight{\ifdim\Gin@nat@height>\textheight\textheight\else\Gin@nat@height\fi}
\makeatother
% Scale images if necessary, so that they will not overflow the page
% margins by default, and it is still possible to overwrite the defaults
% using explicit options in \includegraphics[width, height, ...]{}
\setkeys{Gin}{width=\maxwidth,height=\maxheight,keepaspectratio}
% Set default figure placement to htbp
\makeatletter
\def\fps@figure{htbp}
\makeatother
\setlength{\emergencystretch}{3em} % prevent overfull lines
\providecommand{\tightlist}{%
  \setlength{\itemsep}{0pt}\setlength{\parskip}{0pt}}
\setcounter{secnumdepth}{-\maxdimen} % remove section numbering
\usepackage{setspace}
\usepackage{float}
\usepackage{mathtools}
\usepackage{natbib}
\usepackage[linesnumbered,ruled,vlined]{algorithm2e}
\setcitestyle{numbers,square,comma}
\usepackage{verbatim}
\usepackage{amsthm}
\usepackage{comment}
\ifluatex
  \usepackage{selnolig}  % disable illegal ligatures
\fi
\usepackage[]{natbib}
\bibliographystyle{plainnat}

\title{Clustering of Distributions on 1-Dimensional Manifolds}
\author{}
\date{\vspace{-2.5em}}

\begin{document}
\maketitle

\hypertarget{setup}{%
\section{Setup}\label{setup}}

Suppose we have two manifolds \(\mathcal{M}_1\) and \(\mathcal{M}_2\)
\(\in \mathbb{R}^d\), each of length 1, defined by \(f_1(t)\) and
\(f_2(t)\) respectively (\(f_i : [0, 1] \mapsto \mathbb{R}^d\)). Define
\(\delta\) as the minimum distance between the two manifolds, i.e.,
\(\delta = \min_t \|f_1(t) - f_2(t)\|\), and let \(\delta > 0\). For
now, restrict each \(f_i\) such that the distance along the manifold
between \(f_i(t)\) and \(f_i(s)\) is equal to the difference between
\(t\) and \(s\) (this also implies that each manifold is of length 1).
We sample \(T_1, ..., T_n \stackrel{iid}{\sim} F\) for continuous \(F\)
with support \([0, 1]\) and use \(f_1\) to map the first \(n_1\) points
to \(\mathcal{M}_1\) and \(f_2\) to map the remaining \(n_2 = n - n_1\)
points to \(\mathcal{M}_2\). Let \(X_i = f_1(T_i)\) and
\(Y_j = f_2(T_j)\). Without loss of generality, assume \(n_1 \leq n_2\).

\hypertarget{theory}{%
\section{Theory}\label{theory}}

Let \(D_i = X_{(i)} - X_{(i-1)}\). Then if \(\max_i D_i < \delta\), we
have sufficient separation of points in \(\mathcal{M}_1\).

\hypertarget{uniform-case}{%
\subsection{Uniform case}\label{uniform-case}}

It can be shown that if \(X_i \stackrel{iid}{\sim} Uniform(0, 1)\), then
\(D_i \sim Beta(1, n)\). Therefore,
\(P(\max_i D_i < \delta) \geq (P(D_i < \delta))^n = (1 - (1 - \delta)^n)^n\),
which is a decreasing function for sufficiently large \(n\). This gives
us the result \(\max_i D_i \stackrel{a.s.}{\to} 0\).

\hypertarget{general-case}{%
\subsection{General case}\label{general-case}}

\hypertarget{proof-sketch-that-pd_i-leq-delta-to-1}{%
\subsubsection{\texorpdfstring{Proof sketch that
\(P(D_i \leq \delta) \to 1\)}{Proof sketch that P(D\_i \textbackslash leq \textbackslash delta) \textbackslash to 1}}\label{proof-sketch-that-pd_i-leq-delta-to-1}}

If \(F\) is absolutely continuous, then it can be shown that

\[P(D_i \leq \delta) = 1 - \int_0^{1-\delta} \frac{n!}{(n-i+1)! (i-2)!} (F(x))^{i-2} (1 - F(x + \delta))^{n-i} f(x) dx\]

Making the approximation \(F(x+\delta) \approx F(x) + \delta f(x)\) and
bounding \(f(x) \geq a > 0\), we get:

\[P(D_i \leq \delta) \geq 1 - \int_0^{1-\delta} \frac{n!}{(n-i+1)! (i-2)!} (F(x))^{i-2} (1 - F(x) - a \delta)^{n-i} f(x) dx\]

Then making the substitution \(u = F(x) \implies du = f(x) dx\), we get

\[1 - \int_0^{F(1-\delta)} \frac{n!}{(n-i+1)! (i-2)!} u^{i-2} (1 - u - a \delta)^{n-i} du\]

Which becomes \(1 - C P(n) \delta^{n}\) for some \(C > 0\) and \(P(n)\)
is a polynomial of \(n\). This goes to 0 as \(n \to \infty\), giving us
the result \(D_i \stackrel{p}{\to} 0\).

\hypertarget{computational-results}{%
\section{Computational Results}\label{computational-results}}

TBD

  \bibliography{proposal.bib}

\end{document}
