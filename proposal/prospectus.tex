% Options for packages loaded elsewhere
\PassOptionsToPackage{unicode}{hyperref}
\PassOptionsToPackage{hyphens}{url}
\PassOptionsToPackage{dvipsnames,svgnames*,x11names*}{xcolor}
%
\documentclass[
  11pt,
]{article}
\usepackage{lmodern}
\usepackage{amssymb,amsmath}
\usepackage{ifxetex,ifluatex}
\ifnum 0\ifxetex 1\fi\ifluatex 1\fi=0 % if pdftex
  \usepackage[T1]{fontenc}
  \usepackage[utf8]{inputenc}
  \usepackage{textcomp} % provide euro and other symbols
\else % if luatex or xetex
  \usepackage{unicode-math}
  \defaultfontfeatures{Scale=MatchLowercase}
  \defaultfontfeatures[\rmfamily]{Ligatures=TeX,Scale=1}
\fi
% Use upquote if available, for straight quotes in verbatim environments
\IfFileExists{upquote.sty}{\usepackage{upquote}}{}
\IfFileExists{microtype.sty}{% use microtype if available
  \usepackage[]{microtype}
  \UseMicrotypeSet[protrusion]{basicmath} % disable protrusion for tt fonts
}{}
\makeatletter
\@ifundefined{KOMAClassName}{% if non-KOMA class
  \IfFileExists{parskip.sty}{%
    \usepackage{parskip}
  }{% else
    \setlength{\parindent}{0pt}
    \setlength{\parskip}{6pt plus 2pt minus 1pt}}
}{% if KOMA class
  \KOMAoptions{parskip=half}}
\makeatother
\usepackage{xcolor}
\IfFileExists{xurl.sty}{\usepackage{xurl}}{} % add URL line breaks if available
\IfFileExists{bookmark.sty}{\usepackage{bookmark}}{\usepackage{hyperref}}
\hypersetup{
  pdfauthor={John Koo},
  colorlinks=true,
  linkcolor=Maroon,
  filecolor=Maroon,
  citecolor=Blue,
  urlcolor=blue,
  pdfcreator={LaTeX via pandoc}}
\urlstyle{same} % disable monospaced font for URLs
\usepackage[margin=1in]{geometry}
\usepackage{graphicx,grffile}
\makeatletter
\def\maxwidth{\ifdim\Gin@nat@width>\linewidth\linewidth\else\Gin@nat@width\fi}
\def\maxheight{\ifdim\Gin@nat@height>\textheight\textheight\else\Gin@nat@height\fi}
\makeatother
% Scale images if necessary, so that they will not overflow the page
% margins by default, and it is still possible to overwrite the defaults
% using explicit options in \includegraphics[width, height, ...]{}
\setkeys{Gin}{width=\maxwidth,height=\maxheight,keepaspectratio}
% Set default figure placement to htbp
\makeatletter
\def\fps@figure{htbp}
\makeatother
\setlength{\emergencystretch}{3em} % prevent overfull lines
\providecommand{\tightlist}{%
  \setlength{\itemsep}{0pt}\setlength{\parskip}{0pt}}
\setcounter{secnumdepth}{5}
\usepackage{setspace}
\usepackage{float}
\usepackage{mathtools}
\usepackage{natbib}
\usepackage[linesnumbered,ruled,vlined]{algorithm2e}
\setcitestyle{numbers,square,comma}
\usepackage{verbatim}
\usepackage{amsthm}
\usepackage{comment}
\usepackage[]{natbib}
\bibliographystyle{plainnat}

\title{Community Detection for the\\
Generalized Random Dot Product Graph}
\usepackage{etoolbox}
\makeatletter
\providecommand{\subtitle}[1]{% add subtitle to \maketitle
  \apptocmd{\@title}{\par {\large #1 \par}}{}{}
}
\makeatother
\subtitle{Dissertation Prospectus}
\author{John Koo}
\date{}

\begin{document}
\maketitle

\newcommand{\diag}{\text{diag}}
\newcommand{\tr}{\text{Tr}}
\newcommand{\blockdiag}{\text{blockdiag}}
\newcommand{\indep}{\stackrel{\text{ind}}{\sim}}
\newcommand{\iid}{\stackrel{\text{iid}}{\sim}}
\newcommand{\Bernoulli}{\text{Bernoulli}}
\newcommand{\Betadist}{\text{Beta}}
\newcommand{\BG}{\text{BernoulliGraph}}
\newcommand{\Cat}{\text{Categorical}}
\newcommand{\GRDPG}{\text{GRDPG}}
\newtheorem{definition}{Definition}
\newtheorem{theorem}{Theorem}
\newtheorem{lemma}{Lemma}
\theoremstyle{remark}
\newtheorem*{remark}{Remark}
\theoremstyle{example}
\newtheorem*{example}{Example}

Graph and network data, in which samples are represented not as a
collection of feature vectors but as relationships between pairs of
observations, are increasingly widespread in various fields ranging from
sociology to computer vision. One common goal of analyzing graph data is
community detection or graph clustering, in which the graph is
partitioned into disconnected subgraphs in an unsupervised yet
meaningful manner (e.g., by optimizing an objective function or
recovering unobserved labels). Because traditional clustering techniques
were developed for data that can be represented as vectors, they cannot
be applied directly to graphs. Common examples of such clustering
techniques include \(K\)-means clustering \cite{1056489} and Gaussian
Mixture Models \cite{doi:10.1198/016214502760047131}. In addition, these
techniques assume that the vectors are sampled from a specific type of
distribution (whether implicitly in the case of \(K\)-means clustering
or explicitly in the case of GMM). In this research, we investigate the
use of a family of spectral decomposition based approaches for community
detection in block models (random graph models with inherent community
structure). First, we demonstrate that under the Generalized Random Dot
Product Graph \cite{rubindelanchy2017statistical} framework, all graphs
generated by Block Models (generative graph models with inherent
community structure) can be represented as collections of vectors in
Euclidean space via the Adjacency Spectral Embedding
\cite{athreya2017statistical, lyzinski2014}. Then, noting the particular
structure or configuration that the vectors take, we select appropriate
clustering techniques to apply on the vector representation of the
graph. Finally we derive the asymptotic properties of these methods to
show that our methods are consistent or achieve desirable properties
with high probability. To illustrate this approach, we primarily focus
on the Popularity Adjusted Block Model
\cite{307cbeb9b1be48299388437423d94bf1}.

Our work is essentially about connecting two well-known families of
generative graph models, the Block Model and the (Generalized) Random
Dot Product Graph \cite{rubindelanchy2017statistical} in order to
perform community detection. It is straightforward to show that all
Block Models are various cases of the GRDPG. This fact has been
leveraged to develop community detection methods
\cite{athreya2017statistical, lyzinski2014, rubindelanchy2017statistical}
for the Stochastic Block Model \cite{doi:10.1080/0022250X.1971.9989788}
and Degree Corrected Block Model \cite{Karrer_2011}. Our work begins by
extending this to the Popularity Adjusted Block Model. We show that
under the GRDPG framework, the latent configuration that generates the
PABM consists of orthogonal subspaces. Then we show that two community
detection algorithms arise naturally from this latent configuration:
Orthogonal Spectral Clustering, which is an algorithm of our own design,
and an existing algorithm, Sparse Subspace Clustering \cite{5206547}. We
then extend this work to more general configurations in the latent space
that imply community structure. The overall goal of our research is to
develop a general framework or approach to community detection for a
wide range of Block Models.

\newpage

The overall generative model for inducing community structure can be
described by the following.

Let \((A, X) \sim \text{GRDPG}_{p, q}(F, n)\) such that:

\begin{enumerate}
\def\labelenumi{\arabic{enumi}.}
\tightlist
\item
  Define functions \(\gamma_1, ..., \gamma_K\) such that each
  \(\gamma_k : [0, 1]^r \mapsto \mathbb{R}^d\) and
  \(\gamma_k(t) \neq \gamma_l(t)\) when \(k \neq l\).
\item
  Sample labels
  \(Z_1, ..., Z_n \stackrel{\text{iid}}{\sim}\text{Categorical}(\pi_1, ..., \pi_K)\).
\item
  Sample \(T_1, ..., T_n \stackrel{\text{iid}}{\sim}D\) with support
  \([0, 1]^r\).
\item
  Set latent positions \(X_i = \gamma_{Z_i}(T_i)\) and
  \(X = \begin{bmatrix} X_1 & \cdots & X_n \end{bmatrix}^\top\).
\item
  \(A \sim \text{BernoulliGraph}(X I_{p, q} X^\top)\)
\end{enumerate}

The SBM, DCBM, and PABM are all special cases of this generative model.
Based on the structure of the \(\gamma_k\) functions that correspond to
each of the SBM, DCBM, and PABM, there are clear choices for clustering
algorithms on the embeddings of each of these Block Models. Extensions
of these results may consider which clustering algorithms result in
consistent estimators given restrictions on each of the \(\gamma_k\),
and how to approach community detection when \(\gamma_k\) are unknown
altogether.

The estimated timeline of completion is as follows:

\begin{enumerate}
\def\labelenumi{\arabic{enumi}.}
\item
  Literature review and complete proofs of main theorems: December 2022
\item
  Simulations, real data analyses, R package: March 2022
\item
  Dissertation Completion: April 2022
\end{enumerate}

  \bibliography{proposal.bib}

\end{document}
