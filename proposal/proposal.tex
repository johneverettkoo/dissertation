% Options for packages loaded elsewhere
\PassOptionsToPackage{unicode}{hyperref}
\PassOptionsToPackage{hyphens}{url}
\PassOptionsToPackage{dvipsnames,svgnames*,x11names*}{xcolor}
%
\documentclass[
  11pt,
]{article}
\usepackage{lmodern}
\usepackage{amssymb,amsmath}
\usepackage{ifxetex,ifluatex}
\ifnum 0\ifxetex 1\fi\ifluatex 1\fi=0 % if pdftex
  \usepackage[T1]{fontenc}
  \usepackage[utf8]{inputenc}
  \usepackage{textcomp} % provide euro and other symbols
\else % if luatex or xetex
  \usepackage{unicode-math}
  \defaultfontfeatures{Scale=MatchLowercase}
  \defaultfontfeatures[\rmfamily]{Ligatures=TeX,Scale=1}
\fi
% Use upquote if available, for straight quotes in verbatim environments
\IfFileExists{upquote.sty}{\usepackage{upquote}}{}
\IfFileExists{microtype.sty}{% use microtype if available
  \usepackage[]{microtype}
  \UseMicrotypeSet[protrusion]{basicmath} % disable protrusion for tt fonts
}{}
\makeatletter
\@ifundefined{KOMAClassName}{% if non-KOMA class
  \IfFileExists{parskip.sty}{%
    \usepackage{parskip}
  }{% else
    \setlength{\parindent}{0pt}
    \setlength{\parskip}{6pt plus 2pt minus 1pt}}
}{% if KOMA class
  \KOMAoptions{parskip=half}}
\makeatother
\usepackage{xcolor}
\IfFileExists{xurl.sty}{\usepackage{xurl}}{} % add URL line breaks if available
\IfFileExists{bookmark.sty}{\usepackage{bookmark}}{\usepackage{hyperref}}
\hypersetup{
  colorlinks=true,
  linkcolor=Maroon,
  filecolor=Maroon,
  citecolor=Blue,
  urlcolor=blue,
  pdfcreator={LaTeX via pandoc}}
\urlstyle{same} % disable monospaced font for URLs
\usepackage[margin=1in]{geometry}
\usepackage{graphicx,grffile}
\makeatletter
\def\maxwidth{\ifdim\Gin@nat@width>\linewidth\linewidth\else\Gin@nat@width\fi}
\def\maxheight{\ifdim\Gin@nat@height>\textheight\textheight\else\Gin@nat@height\fi}
\makeatother
% Scale images if necessary, so that they will not overflow the page
% margins by default, and it is still possible to overwrite the defaults
% using explicit options in \includegraphics[width, height, ...]{}
\setkeys{Gin}{width=\maxwidth,height=\maxheight,keepaspectratio}
% Set default figure placement to htbp
\makeatletter
\def\fps@figure{htbp}
\makeatother
\setlength{\emergencystretch}{3em} % prevent overfull lines
\providecommand{\tightlist}{%
  \setlength{\itemsep}{0pt}\setlength{\parskip}{0pt}}
\setcounter{secnumdepth}{5}
\usepackage{setspace}
\usepackage{float}
\usepackage{mathtools}
\usepackage{natbib}
\usepackage[linesnumbered,ruled,vlined]{algorithm2e}
\setcitestyle{numbers,square,comma}
\usepackage{verbatim}
\usepackage{amsthm}
\usepackage{comment}
\usepackage[]{natbib}
\bibliographystyle{plainnat}

\author{}
\date{\vspace{-2.5em}}

\begin{document}


\pagenumbering{gobble}

%\begin{titlepage}
\begin{center}
\LARGE{\textbf{Community Detection Methods for Random Dot Product Graphs and Generalized Random Dot Product Graphs}}\\
\vspace*{2\baselineskip}
\normalsize{A dissertation proposal submitted in partial satisfaction of the requirements for the degree of \\}
Doctor of Philosophy \\
in \\
Statistical Science \\
\vspace*{2\baselineskip}
\Large{John Koo}\\
\vspace*{3\baselineskip}
\Large{\textbf{Research Committee Members}}\\
Dr. Michael Trosset \\
Dr. Minh Tang \\
Dr. Julia Fukuyama \\
Dr. Roni Khardon \\
Dr. Fangzheng Xie \\
\vspace*{3\baselineskip}
Date TBA \\
\vspace*{1\baselineskip}
Department of Statistics \\
Indiana University \\
Bloomington, Indiana \\
\end{center}
% \end{titlepage}

\hypersetup{linkcolor = black}
\newpage
\pagenumbering{roman}
\tableofcontents
\addcontentsline{toc}{section}{\contentsname}

\newpage

\newpage
\pagenumbering{arabic}
\hypersetup{linkcolor = blue}

\newcommand{\diag}{\text{diag}}
\newcommand{\tr}{\text{Tr}}
\newcommand{\blockdiag}{\text{blockdiag}}
\newcommand{\indep}{\stackrel{\text{indep}}{\sim}}
\newcommand{\iid}{\stackrel{\text{iid}}{\sim}}
\newcommand{\Bernoulli}{\text{Bernoulli}}
\newcommand{\Betadist}{\text{Beta}}
\newtheorem{definition}{Definition}
\newtheorem{theorem}{Theorem}
\newtheorem{lemma}{Lemma}
\theoremstyle{remark}
\newtheorem*{remark}{Remark}
\theoremstyle{example}
\newtheorem*{example}{Example}

\hypertarget{introduction}{%
\section{Introduction}\label{introduction}}

\hypertarget{research-goal}{%
\subsection{Research Goal}\label{research-goal}}

Graph and network data have become increasingly widespread in various
fields including sociology, neuroscience, biostatistics, and computer
science. This has resulted in various challenges for researchers who
rely on traditional statistical and machine learning methods, many of
which are incompatible with graph data and instead require the data to
exist as feature vectors in Euclidean space. Such challenges often
involve clustering and community detection. Common clustering methods
typically involve calculating some central or representative point for
each cluster around which the data belonging to that cluster lie (e.g.,
Lloyd's algorithm for \(K\)-means clustering \cite{1056489}, Gaussian
Mixture Models \cite{doi:10.1198/016214502760047131}). Because these
methods involve computing summary statistics within each cluster, such
as the sample average, they cannot be applied directly to graphs,
necessitating methods for transforming the graph data into feature
vectors.

One family of methods for unifying graph community detection with
traditional clustering techniques is Spectral Clustering
\cite{DBLP:journals/corr/abs-0711-0189}, which involves embedding the
graph into Euclidean space, followed by applying a popular clustering
algorithm such as \(K\)-means clustering. The Random Dot Product Graph
(RDPG) \cite{10.1007/978-3-540-77004-6_11} and Generalized Random Dot
Product Graph (GRDPG) \cite{rubindelanchy2017statistical} models take
this further by explicitly constructing generative models such that
latent positions in Euclidean space are used to induce graphs. A
community detection algorithm motivated by this may involve learning the
latent positions given an observed graph and then learning the community
labels given the latent positions.

The aim of our research is to develop consistent community detection
techniques under the RDPG and GRDPG frameworks. First, we explore
existing generative graph models with underlying community structures
that can be inferred by connecting the generative models to the RDPG or
GRDPG. Then we explore other latent structures or mixture distributions
in the latent space that induce graphs for which consistent community
detection is possible.

\hypertarget{notation}{%
\subsection{Notation}\label{notation}}

Let \(G = (V, E)\) be an undirected, unweighted graph with no self-loops
with \(n\) vertices. Denote \(A \in \{0, 1\}^{n \times n}\) as the
adjacency matrix for \(G\) such that \(A_{ij} = 1\) if there exists an
edge between vertices \(i\) and \(j\) and \(A_{ij} = 0\) otherwise.
Because \(G\) is symmetric and contains no self-loops,
\(A_{ij} = A_{ji}\) and \(A_{ii} = 0\) for \(i, j \in [n]\). We further
restrict our analyses to independent Bernoulli graphs. Let
\(P \in [0, 1]^{n \times n}\) be a symmetric matrix of edge
probabilities. Graph \(G\) is sampled from \(P\) by drawing
\(A_{ij} \stackrel{\text{indep}}{\sim}\text{Bernoulli}(P_{ij})\) for
each \(0 \leq i < j \leq n\) (\(A_{ji} = A_{ij}\) and \(A_{ii} = 0\)).
Finally, we denote
\(X = \begin{bmatrix} x_1 & \cdots & x_n \end{bmatrix}^\top \in \mathbb{R}^{n \times d}\)
as the sample \(x_1, ..., x_n \in \mathbb{R}^d\), and denote
\(z_1, ..., z_n \in [K]\) as their corresponding (hidden) labels.

\hypertarget{literature-review}{%
\section{Literature Review}\label{literature-review}}

\hypertarget{generative-graph-models-and-community-detection}{%
\subsection{Generative Graph Models and Community
Detection}\label{generative-graph-models-and-community-detection}}

Generative models for symmetric Bernoulli graphs involve defining the
edge probability matrix \(P\) whose \(ij\)\textsuperscript{th} entry is
the probability of an edge between vertices \(i\) and \(j\) for each
\(i, j \in [n]\). In order to motivate community detection methods, we
restrict the generative model such that for each pair of vertices, the
probability of an edge between the vertices is conditioned on the labels
of the vertices. One such model is the Stochastic Block Model (SBM)
\cite{doi:10.1080/0022250X.1971.9989788}: Given \(K\) communities and
each vertex belonging to one community, the SBM restricts \(P\) to
\(K (K + 1) / 2\) unique entries such that \(P_{ij} = B_{z_i z_j}\) and
\(B \in [0, 1]^{K \times K}\) with entries \(B_{kl}\) denoting the edge
probability of each vertex in community \(k\) having an edge with each
vertex in community \(l\). The homogeneous SBM further restricts \(P\)
to two unique entries such that \(P_{ij} = p\) if \(z_i = z_j\) and
\(P_{ij} = q\) otherwise. Multiple generalizations of the SBM have been
introduced since, including the Degree Corrected Block Model (DCBM) and
the Popularity Adjusted Block Model (PABM). Like the SBM, these models
involve edge probability matrix \(P\) that is restricted to rank
\(d < n\) in part based on community labels.

\hypertarget{generalized-random-dot-product-graphs}{%
\subsection{(Generalized) Random Dot Product
Graphs}\label{generalized-random-dot-product-graphs}}

Like the SBM, DCBM, and PABM, the RDPG and GRDPG are generative models
for graphs involving an edge probability matrix \(P\). The RDPG starts
with points in latent space \(X \in \mathcal{X} \subset \mathbb{R}^d\)
such that \(\forall x, y \in \mathcal{X}\), \(x^\top y \in [0, 1]\).
\(P\) is then constructed as \(P = X X^\top\), and graph \(G\) with
adjacency matrix \(A\) is drawn from \(P\). We provide a more formal
definition of the RDPG and GRDPG below.

\begin{definition}[(Generalized) Random Dot Product Graph]
Let $X \in \mathbb{R}^{n \times d}$ be a collection of $n$ points in $\mathcal{X} \subset \mathbb{R}^d$ such that $\forall x, y \in \mathcal{X}$, $x^\top y.\in [0, 1]$. $G = (V, E)$ is a Random Dot Product Graph if its adjacency matrix $A$ is drawn such that $A_{ij} \sim \text{Bernoulli}(x_i^\top x_j)$ for $i < j$, with $A_{ji} = A_{ij}$ and $A_{ii} = 0$ $\forall i, j \in [n]$. If on the other hand $A_{ij} \sim \text{Bernoulli}(x_i^\top I_{p, q} x_j)$ where $I_{p, q} = \text{blockdiag}(I_p, -I_q)$ and $p + q = d$, then $A$ is the adjacency matrix of a Generalized Random Dot Product Graph. These are denoted by $A \sim \text{RDPG}(X)$ and $A \sim \text{GRDPG}_{p, q}(X)$ respectively.

In addition, let $F$ be a probability distribution with support $\mathcal{X}$, and $x_1, ..., x_n \stackrel{iid}{\sim} F$ with $X = \begin{bmatrix} x_1 & \cdots & x_n \end{bmatrix}^\top$. If $A$ is drawn from $X$ as before, then $(A, X) \sim \text{RDPG}(F, n)$ or $(A, X) \sim \text{GRDPG}_{p, q}(F, n)$. 
\end{definition}

The structure of the RDPG and GRDPG provides a straightforward method
for recovery of the latent positions via spectral embedding.

\begin{definition}[Adjacency Spectral Embedding]
Let $A \sim \text{RDPG}(X)$ for $X \in \mathcal{X} \subset \mathbb{R}^{n \times d}$. Let $A = V \Lambda V^\top$ be the approximate spectral decomposition of $A$ corresponding to the $d$ largest eigenvalues and their corresponding eigenvectors. Then the rows of $V \Lambda^{1/2}$ are the scaled Adjacency Spectral Embedding (ASE) of $A$, and the rows of $V$ are the unscaled ASE of $A$. 

If $A \sim \text{GRDPG}_{p, q}(X)$, then let $A = V \Lambda V^\top$ be the approximate spectral decomposition of $A$ corresponding to the $p$ most positive and $q$ most negative eigenvalues of $A$ and their corresponding eigenvectors. Then the rows of $V |\Lambda|^{1/2}$ and $V$ are the scaled and unscaled ASE of $A$ respectively.
\end{definition}

\citet{athreya2017statistical} showed that under mild conditions, if
\((A_n, X_n) \sim \text{RDPG}(F, n)\) and \(\hat{X}_n\) is the scaled
ASE of \(A_n\), for some sequence of orthogonal matrices \(W_n\),

\begin{equation}
\max_i \|(\hat{X}_n)_i - W_n (X_n)_i \| \stackrel{a.s.}{\to} 0
\end{equation}

Similarly, \citet{rubindelanchy2017statistical} showed that for
\((A_n, X_n) \sim \text{GRDPG}_{p, q}(F, n)\),

\begin{equation}
\max_i \|(\hat{X}_n)_i - Q_n (X_n)_i \| \stackrel{a.s.}{\to} 0
\end{equation}

where \(Q_n\) is a sequence of matrices in \(\mathbb{O}(p, q)\), the
indefinite orthogonal group of order \(p, q\).

It is straightforward to show that all Bernoulli graphs with positive
semidefinite \(P\) are special cases of the RDPG, which is a special
case of the GRDPG, and all graphs generated by \(P\) are special cases
of the GRDPG. This includes the SBM, DCBM, and PABM. In the following
example, we show that the ASE of the SBM has a very particular form.

\begin{example}[Connecting the SBM to the RDPG]
Let $G = (V, E)$ with adjacency matrix $A$ be sampled from the homogeneous SBM with two communities such that within-community edge probability $p$ and between-community edge probability $q$ where $p > q$. Let community 1 have $n_1$ vertices and community 2 have $n_2$ vertices such that $n_1 + n_2 = n$. Without loss of generality, organize $P$ and $A$ such that the $kl^{th}$ block represents edges between communities $k$ and $l$. Then $P = \begin{bmatrix} P^{(11)} & P^{(12)} \\ P^{(21)} & P^{(22)} \end{bmatrix}$ where each block is a constant value, e.g., $P^{(11)}_{ij} = p$. One RDPG representation of this SBM is:

$$X = \begin{bmatrix} 
\sqrt{p} & 0 \\
\vdots & \vdots \\
\sqrt{p} & 0 \\
\sqrt{r^2 / p} & \sqrt{q - r^2 / p} \\ 
\vdots & \vdots \\
\sqrt{r^2 / p} & \sqrt{q - r^2 / p}
\end{bmatrix}
\in \mathbb{R}^{n \times 2}$$

where the first $n_1$ rows are $\begin{bmatrix} \sqrt{p} & 0 \end{bmatrix}$ and the next $n_2$ rows are $\begin{bmatrix} \sqrt{r^2 / p} & \sqrt{q - r^2 / p} \end{bmatrix}$. Then it can be shown that 

$$P = X X^\top$$
\end{example}

The ASE of the assortative SBM consists of points in \(\mathbb{R}^K\)
that lie near one of \(K\) centers, depending on the community label,
leading to ASE followed by \(K\)-means clustering (or similar, e.g.,
GMM) as a consistent community detection algorithm \cite{lyzinski2014}.
A similar result can be shown for the DCBM and PABM but with different
structures in the ASE.

Given sufficiently large sample size \(n\), the scaled ASE of affinity
matrix \(A\) drawn from a RDPG or GRDPG will asymptotically approach the
original latent positions \(X\) with probability 1, up to a linear
transformation (orthogonal transformation for the RDPG, a composition of
an orthogonal and scale transformations for the GRDPG). Thus if \(X\)
consists of points that lie on subspaces of \(\mathbb{R}^d\), then both
the scaled and unscaled ASE of \(A \sim \text{RDPG}(X)\) or
\(A \sim \text{GRDPG}(X)\) will consist of points that lie near
subspaces, with some noise that almost surely goes to \(0\) as
\(n \to \infty\), motivating ASE followed by SSC as an asymptotically
consistent method for community detection. The Popularity Adjusted Block
Model (PABM) \cite{307cbeb9b1be48299388437423d94bf1} is a generative
graph model with underlying communities such that each community lies on
a subspace. \citet{noroozi2019estimation} showed that SSC is able to
recover the subspaces and therefore perform community detection for the
PABM given \(P = X I_{p, q} X^\top\), the edge probability matrix,
rather than \(A\), the adjacency matrix. Combining the results of
\citeauthor{rubindelanchy2017statistical} and
\citeauthor{jmlr-v28-wang13}, it should be possible to recover the
communities using \(A\) as well.

\hypertarget{manifold-learning}{%
\subsection{Manifold Learning}\label{manifold-learning}}

\citet{trosset2020learning} showed that the ASE of a RDPG can be used to
recover one-dimensional manifolds. Suppose
\(f : [0, 1] \mapsto \mathcal{X}\) such that \(f\) is smooth and
\(\mathcal{X}\) represents a curve or one-dimensional manifold in
\(\mathbb{R}^d\). If \(t_1, ..., t_n \stackrel{iid}{\sim} F\) such that
\(F\) has support \([0, 1]\), the latent positions are \(x_i = f(t_i)\)
with \(y_i\) is its corresponding point in the scaled ASE, and
\(d_{\epsilon}(\cdot, \cdot)\) is the shortest path distance of an
\(\epsilon\)-neighborhood graph. Under certain mild conditions, the
shortest path distances of the \(\epsilon\)-neighborhood graph of the
ASE approaches the arc lengths along \(f\):

\begin{equation}
d_{\epsilon}(y_i, y_j) \stackrel{p}{\to} \int_{t_i}^{t_j} \sqrt{\sum_r^d \Big( \frac{d f_r}{d t} \Big)^2} dt
\end{equation}

\citet{athreya2020estimation} extended this further by generating a RDPG
from a mixture of distributions on a curve. In their example, points
were sampled from a mixture of two Beta distributions on the
Hardy-Weinberg curve to construct the latent positions of a RDPG, with
the goal of recovering the hidden mixture distribution from an observed
graph.

\hypertarget{proposed-research}{%
\section{Proposed Research}\label{proposed-research}}

\hypertarget{connecting-existing-generative-graph-models-to-the-rdpg-and-grdpg}{%
\subsection{Connecting Existing Generative Graph Models to the RDPG and
GRDPG}\label{connecting-existing-generative-graph-models-to-the-rdpg-and-grdpg}}

As discussed in \S 2.1, all Bernoulli graph models can be expressed as a
RDPG or GRDPG model, and in particular, the (associative) SBM is a RDPG
with a very specific structure in the latent space. A similar result has
been shown for the DCBM. We will now extend this to the PABM.

\hypertarget{popularity-adjusted-block-model}{%
\subsubsection{Popularity Adjusted Block
Model}\label{popularity-adjusted-block-model}}

We first define the PABM.

\begin{definition}[Popularity Adjusted Block Model]
\label{pabm}
Let $P \in [0, 1]^{n \times n}$ be a symmetric edge probability matrix for a 
set of $n$ 
vertices, $V$. Each vertex has a community label $1, ..., K$, and the rows and 
columns of $P$ are arranged by community label such that $n_k \times n_l$ block 
$P^{(kl)}$ describes the edge probabilities between vertices in communities 
$k$ and $l$ ($P^{(lk)} = (P^{(kl)})^\top$). 
Let graph $G = (V, E)$ be an undirected, unweighted graph such 
that its corresponding adjacency matrix $A \in \{0, 1\}^{n \times n}$ is a 
realization of $\text{Bernoulli}(P)$, i.e., 
$A_{ij} \stackrel{\text{indep}}{\sim}\text{Bernoulli}(P_{ij})$ for $i > j$ 
($A_{ij} = A_{ji}$ and $A_{ii} = 0$). 

If each block $P^{(kl)}$ can be written as the outer product of two vectors:

\begin{equation} \label{eq:pabm}
  P^{(kl)} = \lambda^{(kl)} (\lambda^{(lk)})^\top
\end{equation}

for a set of $K^2$ fixed vectors $\{\lambda^{(st)}\}_{s, t = 1}^K$ where each 
$\lambda^{(st)}$ is a column vector 
of dimension $n_s$, then graph $G$ and its corresponding adjacency matrix $A$ 
is a realization of a popularity adjusted block model with parameters 
$\{\lambda^{(st)}\}_{s, t = 1}^K$. 
\end{definition}

We will use the notation \(A \sim \text{PABM}(\{\lambda^{(kl)}\}_K)\) to
denote a random adjacency matrix \(A\) drawn from a PABM with parameters
\(\lambda^{(kl)}\) consisting of \(K\) underlying communities.

It is trivial to show that the PABM, as well as all graphs such that the
adjacency matrix is drawn such that \(A_{ij} \sim Bernoulli(P_{ij})\),
is a special case of the GRDPG. It can also be shown that the latent
positions of the PABM under the GRDPG framework consists of \(K\)
\(K\)-dimensional subspaces in \(\mathbb{R}^{K^2}\). While there is no
unique latent configuration \(X\) such that \(X X^\top = P\), the edge
probability \(P\) for the PABM, they all have this subspace structure,
and one in particular consists of \emph{orthogonal} subspaces.

\begin{theorem}[Connecting the PABM to the GRDPG for $K = 2$]
\label{theorem1}  
Let 

$$X = \begin{bmatrix} 
\lambda^{(11)} & \lambda^{(12)} & 0 & 0 \\ 
0 & 0 & \lambda^{(21)} & \lambda^{(22)} 
\end{bmatrix}$$

$$U = \begin{bmatrix} 1 & 0 & 0 & 0 \\
0 & 0 & 1 / \sqrt{2} & 1 / \sqrt{2} \\
0 & 0 & 1 / \sqrt{2} & - 1 / \sqrt{2} \\
0 & 1 & 0 & 0 \end{bmatrix}$$

where each $\lambda^{(kl)}$ is a vector as in Definition 1. 
Then $A \sim GRDPG_{3, 1}(X U)$ and $A \sim PABM(\{(\lambda^{(kl)}\}_2)$ are 
equivalent.
\end{theorem}

\begin{theorem}[Generalization to $K > 2$] 
\label{theorem2}
There exists a block diagonal matrix 
$X \in \mathbb{R}^{n \times K^2}$ defined by PABM parameters 
$\{\lambda^{(kl)}\}_K$ and orthonormal matrix 
$U \in \mathbb{R}^{K^2 \times K^2}$ that is fixed 
for each $K$ such that $A \sim GRDPG_{K (K+1) / 2, K (K-1) / 2}(XU)$ and 
$A \sim PABM(\{(\lambda^{(kl)}\})_K)$ are equivalent.
\end{theorem}

\begin{proof}
Define the following matrices from $\{\lambda^{(kl)}\}_K$: 

$$\Lambda^{(k)} = 
\begin{bmatrix} \lambda^{(k,1)} & \cdots & \lambda^{(k, K)} \end{bmatrix}
\in \mathbb{R}^{n_k \times K}$$

\begin{equation} \label{eq:xy}
X = \text{blockdiag}(\Lambda^{(1)}, ..., \Lambda^{(K)}) \in \mathbb{R}^{n \times K^2}
\end{equation}

$$L^{(k)} = \text{blockdiag}(\lambda^{(1k)}, ..., \lambda^{(Kk)}) \in 
\mathbb{R}^{n \times K}$$

$$Y = \begin{bmatrix} L^{(1)} & \cdots & L^{(K)} \end{bmatrix} \in 
\mathbb{R}^{n \times K^2}$$

Then $P = X Y^\top$.

Similar to the $K = 2$ case, we have $Y = X \Pi$ for a permutation matrix
$\Pi$, resulting in $P = X \Pi X^\top$.  
The permutation described by $\Pi$ has $K$ fixed points, which correspond to 
$K$ eigenvalues equal to $1$ with corresponding eigenvectors $e_k$ where 
$k = r (K + 1) + 1$ for $r = 0, ..., K - 1$. It also has 
$\binom{K}{2} = K (K - 1) / 2$ cycles of order $2$. Each cycle corresponds to 
a pair of eigenvalues $+1$ and $-1$ and a pair of eigenvectors 
$(e_s + e_t) / \sqrt{2}$ and $(e_s - e_t) / \sqrt{2}$.

Then $\Pi$ has $K (K + 1) / 2$ eigenvalues equal to $1$ and $K (K - 1) / 2$ 
eigenvalues equal to $-1$. $\Pi$ has the decomposed form 

\begin{equation} \label{eq:permutation}
\Pi = U I_{K (K + 1) / 2, K (K - 1) / 2} U^\top
\end{equation}

The edge probability matrix then can be written as:

\begin{equation} \label{eq:pabm-grdpg}
P = X U I_{p, q} (X U)^\top
\end{equation}

\begin{equation} \label{eq:p}
p = K (K + 1) / 2
\end{equation}

\begin{equation} \label{eq:q}
q = K (K - 1) / 2
\end{equation}

and we can describe the PABM with $K$ communities as a GRDPG with latent 
positions $X U$ with signature $\big( K (K + 1) / 2, K (K - 1) / 2 \big)$.
\end{proof}

\begin{example}[$K = 3$] Using the same notation as in Theorem \ref{theorem2}:

$$X = \begin{bmatrix} 
\lambda^{(11)} & \lambda^{(12)} & \lambda^{(13)} & 0 & 0 & 0 & 0 & 0 & 0 \\
0 & 0 & 0 & \lambda^{(21)} & \lambda^{(22)} & \lambda^{(23)} & 0 & 0 & 0 \\
0 & 0 & 0 & 0 & 0 & 0 & \lambda^{(31)} & \lambda^{(32)} & \lambda^{(33)}
\end{bmatrix}$$

$$Y = \begin{bmatrix} 
\lambda^{(11)} & 0 & 0 & \lambda^{(12)} & 0 & 0 & \lambda^{(13)} & 0 & 0 \\
0 & \lambda^{(21)} & 0 & 0 & \lambda^{(22)} & 0 & 0 & \lambda^{(23)} & 0 \\
0 & 0 & \lambda^{(31)} & 0 & 0 & \lambda^{(32)} & 0 & 0 & \lambda^{(33)}
\end{bmatrix}$$

Then $P = X Y^\top$ and $Y = X \Pi$ where $\Pi$ is a permutation matrix 
consisting of $3$ fixed points and $3$ cycles of order 2:

$$\Pi = \begin{bmatrix} 
1 & 0 & 0 & 0 & 0 & 0 & 0 & 0 & 0 \\
0 & 0 & 0 & 1 & 0 & 0 & 0 & 0 & 0 \\
0 & 0 & 0 & 0 & 0 & 0 & 1 & 0 & 0 \\
0 & 1 & 0 & 0 & 0 & 0 & 0 & 0 & 0 \\
0 & 0 & 0 & 0 & 1 & 0 & 0 & 0 & 0 \\
0 & 0 & 0 & 0 & 0 & 0 & 0 & 1 & 0 \\
0 & 0 & 1 & 0 & 0 & 0 & 0 & 0 & 0 \\
0 & 0 & 0 & 0 & 0 & 1 & 0 & 0 & 0 \\
0 & 0 & 0 & 0 & 0 & 0 & 0 & 0 & 1
\end{bmatrix}$$

* Positions 1, 5, 9 are fixed.

* The cycles of order 2 are $(2, 4)$, $(3, 7)$, and $(6, 8)$.
    
Therefore, we can decompose $\Pi = U I_{6, 3} U^\top$ where the first three 
columns of $U$ consist of $e_1$, $e_5$, and $e_9$ corresponding to the fixed 
positions $1$, $5$, and $9$, the next three columns consist of eigenvectors 
$(e_k + e_l) / \sqrt{2}$, and the last three columns consist of eigenvectors 
$(e_k - e_l) / \sqrt{2}$, where pairs $(k, l)$ correspond to the cycles of 
order 2 described above.

The latent positions are the rows of  
$$XU = \begin{bmatrix}
  \lambda^{(11)} & 0 & 0 & 
  \lambda^{(12)} / \sqrt{2} & \lambda^{(13)} / \sqrt{2} & 0 & 
  \lambda^{(12)} / \sqrt{2} & \lambda^{(13)} / \sqrt{2} & 0 \\
  0 & \lambda^{(22)} & 0 & 
  \lambda^{(21)} / \sqrt{2} & 0 & \lambda^{(23)} / \sqrt{2} & 
  -\lambda^{(21)} / \sqrt{2} & 0 & \lambda^{(23)} / \sqrt{2} \\
  0 & 0 & \lambda^{(33)} & 
  0 & \lambda^{(31)} / \sqrt{2} & \lambda^{(32)} / \sqrt{2} & 
  0 & -\lambda^{(31)} / \sqrt{2} & -\lambda^{(32)} / \sqrt{2}
\end{bmatrix}$$
\end{example}

In Theorem \ref{theorem2}, we showed that a possible latent
configuration for a \(K\)-community PABM under the GRDPG framework
consists of latent positions in \(\mathbb{R}^{K^2}\) such that each
community corresponds to a \(K\)-dimensional subspace such that each
subspace is orthogonal to the others.

\begin{theorem}
\label{osc}
Let $P = V \Lambda V^\top$ be the spectral decomposition of the edge probability matrix of a PABM. Define $B = n V V^\top$. Then $B_{ij} = 0$ $\forall i, j$ in different communities. 
\end{theorem}

If \(\hat{V}\) is the \emph{unscaled} ASE of \(A\), then results from
\citeauthor{rubindelanchy2017statistical} imply
\(n \hat{V} \hat{V}^\top \stackrel{a.s.}{\to} n V V^\top\). Then for
vertices \(i\) and \(j\) belonging to separate communities, the
\(ij\)\textsuperscript{th} entry of \(n \hat{V} \hat{V}^\top\)
approaches 0 with probability 1:

\begin{theorem}
\label{theorem4} 
Let $\hat{B}_n$ with entries $\hat{B}_n^{(ij)}$ be the affinity matrix from OSC 
(Alg. 1). Then $\forall$ pairs $(i, j)$ belonging to different communities 
and sparsity factor satisfying $n \rho_n = \omega\{(\log n)^{4c}\}$, 

\begin{equation} \label{eq:thm4}
\max_{i, j} |n (\hat{v}_n^{(i)})^\top \hat{v}_n^{(j)}| = 
O_P \Big( \frac{(\log n)^c}{\sqrt{n \rho_n}} \Big)
\end{equation}

This provides the result that $\forall i, j$ in different communities, 
$\hat{B}_n^{(ij)} \stackrel{a.s.}{\to} 0$.
\end{theorem}

\begin{algorithm}[t]
  \DontPrintSemicolon
  \SetAlgoLined
  \KwData{Adjacency matrix $A$, number of communities $K$}
  \KwResult{Community assignments $1, ..., K$}
    Compute the eigenvectors of $A$ that correspond to the $K (K+1) / 2$ most 
    positive eigenvalues and $K (K-1) / 2$ most negative eigenvalues. Construct 
    $V$ using these eigenvectors as its columns.\;
    Compute $B = |n V V^\top|$, applying $|\cdot|$ entry-wise.\;
    Construct graph $G$ using $B$ as its similarity matrix.\;
    Partition $G$ into $K$ disconnected subgraphs  
    (e.g., using edge thresholding or spectral clustering).\;
    Map each partition to the community labels $1, ..., K$.\;
  \caption{Orthogonal Spectral Clustering.}
\end{algorithm}

In addition to OSC, since the communities are subspaces in the latent
space, we can leverage existing methods for subspace clustering. Of
particular interest is Sparse Subspace Clustering (SSC), which is
performed by solving an optimization problem for each observed point in
a sample. Given \(X \in \mathbb{R}^{n \times d}\) with vectors
\(x_i^\top \in \mathbb{R}^d\) as rows of \(X\), the optimization problem
\(c_i = \min\limits_{c} \|c\|_1\) subject to \(x_i = X_{-i} c\) and
\(c^{(i)} = 0\) is solved for each \(i \in [n]\). The solutions are
collected into matrix
\(C = \begin{bmatrix} c_1 & \cdots & c_n \end{bmatrix}^\top\) to
construct affinity matrix \(B = |C| + |C^\top|\). If each \(x_i\) lie
perfectly on one of \(K\) subspaces, \(B\) is sparse such that
\(B_{ij} = 0\) \(\forall x_i, x_j\) belonging to different subspaces.
Then \(B\) can describe a graph with at least \(K\) disjoint subgraphs,
and if the number of subgraphs is exactly \(K\), each subgraph maps onto
a subspace.

In practice, SSC is performed by solving the LASSO problems:

\begin{equation} \label{eq:ssc}
c_i = \arg\min_c \frac{1}{2} \|x_i - X_{-i} c \|_2^2 + \lambda \|c\|_1
\end{equation}

for some sparsity parameter \(\lambda > 0\). The \(c_i\) vectors are
then collected into \(C\) and \(B\) as described before. If \(X\) is
noisy such that each \(x_i\) does not lie exactly on one of \(K\)
subspaces but near it, the choice of \(\lambda\) becomes important in
guaranteeing the Subspace Detection Property (SDP)
\cite{jmlr-v28-wang13}.

\begin{definition}[Subspace Detection Property] 
Let $X = \begin{bmatrix} x_1 & \cdots & x_n \end{bmatrix}^\top$ be noisy 
points sampled from $K$ subspaces. Let $C$ and $B$ be constructed from the 
solutions of LASSO problems as described in (\ref{eq:ssc}). If each column of 
$C$ has nonzero norm and $B_{ij} = 0$ $\forall$ $x_i$ and $x_j$ sampled from 
different subspaces, then $X$ obeys the Subspace Detection Property. 
\end{definition}

\begin{remark} 
In practice, a noisy sample $X$ often does not obey SDP. 
In such cases, $B$ is treated as an affinity matrix for a graph which 
is then partitioned into $K$ subgraphs to obtain the clustering. On the other 
hand, if $X$ does obey the SDP, $B$ describes a graph 
with at least $K$ disconnected subgraphs. Ideally, when SDP holds, 
there are exactly $K$ subgraphs which map to each subspace, 
but it could be the case that some of the subspaces are represented by 
multiple disconnected subgraphs. SDP is contingent 
on choosing a sufficiently large sparsity parameter $\lambda$. 
\end{remark}

Since every ASE of the PABM consists of subspaces and as
\(n \to \infty\) each point of the ASE approaches its subspace almost
surely, SSC should also work for PABM community detection. Combining
this with the results by \citeauthor{jmlr-v28-wang13}, which state that
if the points lie sufficiently close to their respective subspaces and
the cosine of the angles between subspaces is sufficiently small, SDP
will hold. We show that the unscaled ASE exhibits exactly these
conditions for sufficiently large \(n\).

\begin{theorem}
\label{theorem5}
Let $P_n$ describe the edge probability matrix of the PABM with 
$n$ vertices, and let $A_n \sim \text{Bernoulli}(P_n)$.  Let $\hat{V}_n$ be the 
matrix of eigenvectors of $A_n$ corresponding to the $K (K + 1) / 2$ most
positive and $K (K - 1) / 2$ most negative eigenvalues. Then 
$\exists \lambda > 0$ and $N \in \mathbb{N}$ such that when $n > N$, 
$\sqrt{n} \hat{V}_n$ obeys the subspace detection property with probability 1.
\end{theorem}

\hypertarget{proposed-generative-graph-models}{%
\subsection{Proposed Generative Graph
Models}\label{proposed-generative-graph-models}}

In the previous sections, we connected well-known and highly structured
generative graph models to the RDPG and GRDPG to show that highly
structured latent configurations generate graphs consistent with these
models: The latent space for the SBM consists of \(K\) point masses, the
latent space for the DCBM consists of \(K\) rays emanating from the
origin, and the latent space for the PABM consists of \(K\)
\(K\)-dimensional subspaces. In the following sections, we explore
additional structured latent configurations corresponding to community
structure and develop methods for community detection based on the
consistency of the ASE and the structural forms of the latent
configurations.

The general structure of interest can be described as follows: Suppose
that in the latent space \(\mathcal{X} \subset \mathbb{R}^d\), sample
\(X\) of \(n\) points lie on a union of \(K\) disjoint manifolds with
each manifold corresponding to a community. If
\(A \sim \text{RDPG}(X)\), we wish to recover the community labels (up
to permutation) from \(A\).

Equivalently, suppose that probability distribution \(F\) is described
as follows:

\begin{enumerate}
\def\labelenumi{\arabic{enumi}.}
\tightlist
\item
  Define functions \(f_1, ..., f_K\) such that
  \(f_k : [0, 1] \mapsto \mathcal{X}\) and \(f_k(t) \neq f_l(t)\)
  \(\forall k, l \in [K]\) and \(t \in [0, 1]\).
\item
  Sample labels
  \(z_1, ..., z_n \stackrel{\text{iid}}{\sim}Categorical(\pi_1, ..., \pi_K)\).
\item
  Sample \(t_1, ..., t_n \stackrel{\text{iid}}{\sim}D\) where \(D\) has
  support \([0, 1]\).
\item
  Set latent positions \(x_i = f_{z_i}(t_i)\) and
  \(X = \begin{bmatrix} x_1 & \cdots & x_n \end{bmatrix}^\top\).
\end{enumerate}

Then if \((A, X) \sim \text{RDPG}(F, n)\) and we observe \(A\), we wish
to recover hidden labels \(z_1, ..., z_n\).

\hypertarget{affine-segments}{%
\subsubsection{Affine Segments}\label{affine-segments}}

In this section, we will consider latent positions sampled uniformly
from parallel unit length segments.

\begin{example}
Let $U_1, ..., U_{n_1}, V_1, ..., V_{n_2} \stackrel{\text{iid}}{\sim}Uniform(0, 1)$, $f_1(t) = (t, 0)$, and $f_2(t) = (t, a)$. $X_i = f_1(U_i)$ and $Y_j = f_2(V_j)$ $\forall i \in [n_1]$ and $j \in [n_2]$. If we observe $X_1, ..., X_{n_1}, Y_1, ..., Y_{n_2}$, what approach will allow us to group the observations by segment?
\end{example}

We will motivate an approach by the following example.

\begin{example}
Let $U_1, ..., U_n \stackrel{\text{iid}}{\sim}Uniform(0, 1)$ with order statistics 
$U_{(1)}, ..., U_{(n)}$. Then $\forall a \in (0, 1)$ and $\delta \in (0, 1)$, $\exists N = N(\delta, a) < \infty$ such that $\forall n \geq N$, 

\begin{equation}
P(\max_i U_{(i+1)} - U_{(i)} \leq a) \geq 1 - \delta / 2
\end{equation}

Where $N(\delta, a)$ is monotone increasing with respect to $\delta$ and $a$. To prove this, we start with the fact that $U_{(i+1)} - U_{(i)} \sim Beta(1, n)$. Then 

\begin{equation}
P(U_{(i+1)} - U_{(i)} \leq a) = 1 - (1 - a)^n
\end{equation}

and 

\begin{equation}
\label{eq:unsolvable}
P(\max_i U_{(i+1)} - U_{(i)} \leq a) \geq (P(U_{(i+1)} - U_{(i)} \leq a))^n = (1 - (1 - a)^n)^n
\end{equation}

This expression is monotone increasing $\forall n \geq N_1$ for some $N_1 < \infty$. 
Setting $(1 - (1 - a)^{N_2})^{N_2} \geq 1 - \delta / 2$, we can solve for a finite ${N_2}$. Then $N = \max(N_1, N_2)$.
\end{example}

If we extend this example such that \(n_1\) points are sampled uniformly
from the segment \(f_1(t) = (t, 0)\) and \(n_2\) points are sampled
uniformly from the segment \(f_2(t) = (t, a)\) for \(t \in [0, 1]\),
then a sample of size \(N(\delta, a)\) is sufficient to satisfy:

\begin{equation}
P(\max_i X_{(i+1)} - X_{(i)} \leq a \leq \min_{i, j} \|X_i - Y_j\|) 
\geq 1 - \delta / 2
\end{equation}

and similar for
\(P(\max_j Y_{(j+1)} - Y_{(j)} \leq \min_{i, j} \|X_i - Y_j\|)\), for
\(X_i\) in the first segment and \(Y_j\) in the second segment and
\(X_{(i)}\), \(Y_{(j)}\) are order statistics in the first coordinate.
If each segment corresponds to a community, this leads to the following
two results:

\begin{enumerate}
\def\labelenumi{\arabic{enumi}.}
\item
  Single linkage clustering with \(K = 2\) will perform perfect
  community detection with probability at least \(1 - \delta\).
\item
  An \(\epsilon\)-neighborhood graph with \(\epsilon \in (0, a)\) will
  consists of at least 2 disjoint subgraphs such that no subgraph
  consists of members of two different communities (analogous to the
  SDP), with probability at least \(1 - \delta\).
\end{enumerate}

We can then further extend this to the case where points are drawn from
unit segments with noise.

\hypertarget{mainfold-learning}{%
\subsubsection{Mainfold Learning}\label{mainfold-learning}}

If instead of sampling uniformly from line segments of unit length, we
sample uniformly from a 1 dimensional manifolds of unit length, the
above property still holds. Let
\(U_1, ..., U_n \stackrel{iid}{\sim} Uniform(0, 1)\) and
\(f : [0, 1] \mapsto \mathbb{R}^d\) be a smooth function such that
\(\int_u^v \sqrt{\sum_k (df_k / dt)^2} dt = \|u - v\|\). Then
\(U_{(i+1)} - U_{(i)} \geq \|f(U_{(i+1)}) - f(U_{(i)})\|\), so
\(P(U_{(i+1)} - U_{(i)} \leq \alpha) \leq P(\|f(U_{(i+1)}) - f(U_{(i)})\| \leq a)\).
If the shortest distance between the two manifolds defined by \(f_1\)
and \(f_2\) with the same restriction is \(a\), then the same \(N\) as
before is sufficient, although perhaps a more lenient lower bound can be
derived based on the shape of \(f_k(\cdot)\).

\hypertarget{summary}{%
\section{Summary}\label{summary}}

\hypertarget{estimated-timeline-of-completion}{%
\section{Estimated Timeline of
Completion}\label{estimated-timeline-of-completion}}

Literature review: August 2021

Complete proofs of main theorems: January 2022

Simulations and real data analyses: March 2022

Dissertation completion: April 2022

\newpage

\renewcommand\refname{References}
  \bibliography{proposal.bib}

\end{document}
