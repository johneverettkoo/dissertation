% Options for packages loaded elsewhere
\PassOptionsToPackage{unicode}{hyperref}
\PassOptionsToPackage{hyphens}{url}
\PassOptionsToPackage{dvipsnames,svgnames*,x11names*}{xcolor}
%
\documentclass[
  11pt,
]{article}
\usepackage{lmodern}
\usepackage{amssymb,amsmath}
\usepackage{ifxetex,ifluatex}
\ifnum 0\ifxetex 1\fi\ifluatex 1\fi=0 % if pdftex
  \usepackage[T1]{fontenc}
  \usepackage[utf8]{inputenc}
  \usepackage{textcomp} % provide euro and other symbols
\else % if luatex or xetex
  \usepackage{unicode-math}
  \defaultfontfeatures{Scale=MatchLowercase}
  \defaultfontfeatures[\rmfamily]{Ligatures=TeX,Scale=1}
\fi
% Use upquote if available, for straight quotes in verbatim environments
\IfFileExists{upquote.sty}{\usepackage{upquote}}{}
\IfFileExists{microtype.sty}{% use microtype if available
  \usepackage[]{microtype}
  \UseMicrotypeSet[protrusion]{basicmath} % disable protrusion for tt fonts
}{}
\makeatletter
\@ifundefined{KOMAClassName}{% if non-KOMA class
  \IfFileExists{parskip.sty}{%
    \usepackage{parskip}
  }{% else
    \setlength{\parindent}{0pt}
    \setlength{\parskip}{6pt plus 2pt minus 1pt}}
}{% if KOMA class
  \KOMAoptions{parskip=half}}
\makeatother
\usepackage{xcolor}
\IfFileExists{xurl.sty}{\usepackage{xurl}}{} % add URL line breaks if available
\IfFileExists{bookmark.sty}{\usepackage{bookmark}}{\usepackage{hyperref}}
\hypersetup{
  colorlinks=true,
  linkcolor=Maroon,
  filecolor=Maroon,
  citecolor=Blue,
  urlcolor=blue,
  pdfcreator={LaTeX via pandoc}}
\urlstyle{same} % disable monospaced font for URLs
\usepackage[margin=1in]{geometry}
\usepackage{graphicx,grffile}
\makeatletter
\def\maxwidth{\ifdim\Gin@nat@width>\linewidth\linewidth\else\Gin@nat@width\fi}
\def\maxheight{\ifdim\Gin@nat@height>\textheight\textheight\else\Gin@nat@height\fi}
\makeatother
% Scale images if necessary, so that they will not overflow the page
% margins by default, and it is still possible to overwrite the defaults
% using explicit options in \includegraphics[width, height, ...]{}
\setkeys{Gin}{width=\maxwidth,height=\maxheight,keepaspectratio}
% Set default figure placement to htbp
\makeatletter
\def\fps@figure{htbp}
\makeatother
\setlength{\emergencystretch}{3em} % prevent overfull lines
\providecommand{\tightlist}{%
  \setlength{\itemsep}{0pt}\setlength{\parskip}{0pt}}
\setcounter{secnumdepth}{5}
\usepackage{setspace}
\usepackage{float}
\usepackage{mathtools}
\usepackage{natbib}
\usepackage[linesnumbered,ruled,vlined]{algorithm2e}
\setcitestyle{numbers,square,comma}
\usepackage{verbatim}
\usepackage{amsthm}
\usepackage{comment}
\usepackage[]{natbib}
\bibliographystyle{plainnat}

\author{}
\date{\vspace{-2.5em}}

\begin{document}


\pagenumbering{gobble}

%\begin{titlepage}
\begin{center}
\LARGE{\textbf{Community Detection on Subspaces and Manifolds}}\\
\vspace*{2\baselineskip}
\normalsize{A dissertation proposal submitted in partial satisfaction of the requirements for the degree of \\}
Doctor of Philosophy \\
in \\
Statistical Science \\
\vspace*{2\baselineskip}
\Large{John Koo}\\
\vspace*{3\baselineskip}
\Large{\textbf{Research Committee Members}}\\
Dr. Michael Trosset \\
Dr. Minh Tang \\
Dr. Julia Fukuyama \\
Dr. Roni Khardon \\
Dr. Fangzheng Xie \\
\vspace*{3\baselineskip}
Date TBA \\
\vspace*{1\baselineskip}
Department of Statistics \\
Indiana University \\
Bloomington, Indiana \\
\end{center}
% \end{titlepage}

\hypersetup{linkcolor = black}
\newpage
\pagenumbering{roman}
\tableofcontents
\addcontentsline{toc}{section}{\contentsname}

\newpage

\newpage
\pagenumbering{arabic}
\hypersetup{linkcolor = blue}

\newcommand{\diag}{\text{diag}}
\newcommand{\tr}{\text{Tr}}
\newcommand{\blockdiag}{\text{blockdiag}}
\newcommand{\indep}{\stackrel{\text{indep}}{\sim}}
\newcommand{\iid}{\stackrel{\text{iid}}{\sim}}
\newcommand{\Bernoulli}{\text{Bernoulli}}
\newcommand{\Betadist}{\text{Beta}}
\newtheorem{definition}{Definition}
\newtheorem{theorem}{Theorem}
\newtheorem{lemma}{Lemma}
\theoremstyle{remark}
\newtheorem*{remark}{Remark}
\theoremstyle{example}
\newtheorem*{example}{Example}

\hypertarget{research-goal}{%
\section{Research Goal}\label{research-goal}}

Typical clustering methods often work by assuming that each cluster is
represented by some central point around which the data belonging to
that cluster lie Such methods, including Lloyd's algorithm for
\(k\)-means clustering \cite{1056489} and Gaussian Mixture Models
\cite{doi:10.1198/016214502760047131}, involve iteratively updating the
central points by averaging data belonging to each cluster and cluster
memberships by comparing proximity to central points. The aim of our
research is to explore clustering and community detection methods for
which each cluster is represented not by a central point but by
subspaces or manifolds. In addition, we aim to explore various
applications of such methods and connect them to preexisting clustering
and community detection problems.

\hypertarget{literature-review}{%
\section{Literature Review}\label{literature-review}}

\hypertarget{subspace-clustering}{%
\subsection{Subspace Clustering}\label{subspace-clustering}}

Subspace clustering, which assumes that points in \(\mathbb{R}^d\) each
lie on one of \(K\) subspaces of \(\mathbb{R}^d\), is an approach that
has found a wide range of uses by the Statistics and Machine Learning
communities, particularly within the field of Computer Vision
\cite{5206547}.

Of particular interest is Sparse Subspace Clustering (SSC), which is
performed by solving an optimization problem for each observed point in
a sample. Given \(X \in \mathbb{R}^{n \times d}\) with vectors
\(x_i^\top \in \mathbb{R}^d\) as rows of \(X\), the optimization problem
\(c_i = \min\limits_{c} \|c\|_1\) subject to \(x_i = X_{-i} c\) and
\(c^{(i)} = 0\) is solved for each \(i \in [n]\). The solutions are
collected into matrix
\(C = \begin{bmatrix} c_1 & \cdots & c_n \end{bmatrix}^\top\) to
construct affinity matrix \(B = |C| + |C^\top|\). If each \(x_i\) lie
perfectly on one of \(K\) subspaces, \(B\) is sparse such that
\(B_{ij} = 0\) \(\forall x_i, x_j\) belonging to different subspaces.
Then \(B\) can describe a graph with at least \(K\) disjoint subgraphs,
and if the number of subgraphs is exactly \(K\), each subgraph maps onto
a subspace.

In practice, SSC is performed by solving the LASSO problems:

\begin{equation} \label{eq:ssc}
c_i = \arg\min_c \frac{1}{2} \|x_i - X_{-i} c \|_2^2 + \lambda \|c\|_1
\end{equation}

for some sparsity parameter \(\lambda > 0\). The \(c_i\) vectors are
then collected into \(C\) and \(B\) as described before. If \(X\) is
noisy in that each \(x_i\) does not lie exactly on one of \(K\)
subspaces but near it, the choice of \(\lambda\) becomes important in
guaranteeing the Subspace Detection Property (SDP)
\cite{jmlr-v28-wang13}.

\begin{definition}[Subspace Detection Property] 
Let $X = \begin{bmatrix} x_1 & \cdots & x_n \end{bmatrix}^\top$ be noisy 
points sampled from $K$ subspaces. Let $C$ and $B$ be constructed from the 
solutions of LASSO problems as described in (\ref{eq:ssc}). If each column of 
$C$ has nonzero norm and $B_{ij} = 0$ $\forall$ $x_i$ and $x_j$ sampled from 
different subspaces, then $X$ obeys the Subspace Detection Property. 
\end{definition}

\begin{remark} 
In practice, a noisy sample $X$ often does not obey SDP. 
In such cases, $B$ is treated as an affinity matrix for a graph which 
is then partitioned into $K$ subgraphs to obtain the clustering. On the other 
hand, if $X$ does obey the SDP, $B$ describes a graph 
with at least $K$ disconnected subgraphs. Ideally, when SDP holds, 
there are exactly $K$ subgraphs which map to each subspace, 
but it could be the case that some of the subspaces are represented by 
multiple disconnected subgraphs. SDP is contingent 
on choosing a sufficiently large sparsity parameter $\lambda$. 
\end{remark}

\hypertarget{generalized-random-dot-product-graphs}{%
\subsection{(Generalized) Random Dot Product
Graphs}\label{generalized-random-dot-product-graphs}}

It has been shown by \citet{jmlr-v28-wang13} that given sufficiently low
noise in \(X\) and sufficiently low affinity between pairs of subspaces
as measured by the cosine of the angle between subspaces, SDP holds for
a specific range of \(\lambda\). This provides theoretical justification
for applying SSC to Random Dot Product Graphs (RDPG)
\cite{athreya2017statistical} and Generalized Random Dot Product Graphs
(GRDPG) \cite{rubindelanchy2017statistical} for which the data in the
latent space lie on subspaces.

\begin{definition}[(Generalized) Random Dot Product Graph]
Let $X \in \mathbb{R}^{n \times d}$ be a collection of $n$ points in $\mathcal{X} \subset \mathbb{R}^d$ such that $\forall x, y \in \mathcal{X}$, $x^\top y.\in [0, 1]$. $G = (V, E)$ is a Random Dot Product Graph if its adjacency matrix $A$ is drawn such that $A_{ij} \sim \text{Bernoulli}(x_i^\top x_j)$ for $i < j$, with $A_{ji} = A_{ij}$ and $A_{ii} = 0$ $\forall i, j \in [n]$. If on the other hand $A_{ij} \sim \text{Bernoulli}(x_i^\top I_{p, q} x_j)$ where $I_{p, q} = \text{blockdiag}(I_p, -I_q)$ and $p + q = d$, then $A$ is the adjacency matrix of a Generalized Random Dot Product Graph. These are denoted by $A \sim \text{RDPG}(X)$ and $A \sim \text{GRDPG}_{p, q}(X)$ respectively.

In addition, let $F$ be a probability distribution with support $\mathcal{X}$, and $x_1, ..., x_n \stackrel{iid}{\sim} F$ with $X = \begin{bmatrix} x_1 & \cdots & x_n \end{bmatrix}^\top$. If $A$ is drawn from $X$ as before, then $(A, X) \sim \text{RDPG}(F, n)$ or $(A, X) \sim \text{GRDPG}_{p, q}(F, n)$. 
\end{definition}

\begin{definition}[Adjacency Spectral Embedding]
Let $A \sim \text{RDPG}(X)$ for $X \in \mathcal{X} \subset \mathbb{R}^{n \times d}$. Let $A = V \Lambda V^\top$ be the approximate spectral decomposition of $A$ corresponding to the $d$ largest eigenvalues and their corresponding eigenvectors. Then the rows of $V \Lambda^{1/2}$ are the scaled Adjacency Spectral Embedding (ASE) of $A$, and the rows of $V$ are the unscaled ASE of $A$. 

If $A \sim \text{GRDPG}_{p, q}(X)$, then let $A = V \Lambda V^\top$ be the approximate spectral decomposition of $A$ corresponding to the $p$ most positive and $q$ most negative eigenvalues of $A$ and their corresponding eigenvectors. Then the rows of $V |\Lambda|^{1/2}$ and $V$ are the scaled and unscaled ASE of $A$ respectively.
\end{definition}

\citeauthor{athreya2017statistical} showed that under mild conditions,
if \((A_n, X_n) \sim \text{RDPG}(F, n)\), if \(\hat{X}_n\) is the scaled
ASE of \(A_n\), for some sequence of orthogonal matrices \(W_n\),

\begin{equation}
\max_i \|(\hat{X}_n)_i - W_n (X_n)_i \| \stackrel{a.s.}{\to} 0
\end{equation}

Similarly, \citeauthor{rubindelanchy2017statistical} showed that for
\((A_n, X_n) \sim \text{GRDPG}_{p, q}(F, n)\),

\begin{equation}
\max_i \|(\hat{X}_n)_i - Q_n (X_n)_i \| \stackrel{a.s.}{\to} 0
\end{equation}

where \(Q_n\) is a sequence of matrices in \(\mathbb{O}(p, q)\), the
indefinite orthogonal group of order \(p, q\).

Given sufficiently large sample size \(n\), the scaled ASE of affinity
matrix \(A\) drawn from a RDPG or GRDPG will asymptotically approach the
original latent positions \(X\) with probability 1, up to a linear
transformation (orthogonal transformation for the RDPG, a composition of
an orthogonal transformation and scale transformation for the GRDPG).
Thus if \(X\) consists of points that lie on subspaces of
\(\mathbb{R}^d\), then both the scaled and unscaled ASE of
\(A \sim \text{RDPG}(X)\) or \(A \sim \text{GRDPG}(X)\) will consist of
points that lie near subspaces, with some noise that almost surely goes
to \(0\) as \(n \to \infty\), motivating ASE followed by SSC as an
asymptotically consistent method for community detection. The Popularity
Adjusted Block Model (PABM) \cite{307cbeb9b1be48299388437423d94bf1} is a
generative graph model with underlying communities such that each
community lies on a subspace. \citet{noroozi2019estimation} showed that
SSC is able to recover the subspaces and therefore perform community
detection for the PABM given \(P = X I_{p, q} X^\top\), the edge
probability matrix, rather than \(A\), the adjacency matrix. Combining
the results of \citeauthor{rubindelanchy2017statistical} and
\citeauthor{jmlr-v28-wang13}, it should be possible to recover the
communities using \(A\) as well.

\hypertarget{manifold-learning}{%
\subsection{Manifold Learning}\label{manifold-learning}}

In addition to subspaces, \citet{trosset2020learning} showed that the
ASE of a RDPG can be used to recover one-dimensional manifolds. Suppose
\(f : [0, 1] \mapsto \mathcal{X}\) such that \(f\) is smooth and
\(\mathcal{X}\) represents a curve or one-dimensional manifold in
\(\mathbb{R}^d\). If \(t_1, ..., t_n \stackrel{iid}{\sim} F\) such that
\(F\) has support \([0, 1]\), the latent positions are \(x_i = f(t_i)\)
with \(y_i\) is its corresponding point in the scaled ASE, and
\(d_{\epsilon}(\cdot, \cdot)\) is the shortest path distance of an
\(\epsilon\)-neighborhood graph. Under certain mild conditions, the
shortest path distances of the \(\epsilon\)-neighborhood graph of the
ASE approaches the arc lengths along \(f\):

\begin{equation}
d_{\epsilon}(y_i, y_j) \stackrel{p}{\to} \int_{t_i}^{t_j} \sqrt{1 - f'(t)} dt
\end{equation}

This suggests that if the latent positions of a RDPG consists of \(K\)
disjoint manifolds, the \(\epsilon\)-neighborhood graph will consist of
\(K\) disjoint subgraphs as \(\epsilon \to 0\) and \(n \to \infty\).

\hypertarget{proposed-research}{%
\section{Proposed Research}\label{proposed-research}}

\hypertarget{subspace-clustering-1}{%
\subsection{Subspace Clustering}\label{subspace-clustering-1}}

As discussed in \S 2.2, if the latent positions of a RDPG or GRDPG model
are such that they lie on a small number of subspaces, ASE followed by
SSC may be able to identify whether vertices of the graph \(v_i, v_j\)
belong to the same subspace or to different subspaces, and one such
model that is consistent with this construction is the PABM.

\hypertarget{popularity-adjusted-block-model}{%
\subsubsection{Popularity Adjusted Block
Model}\label{popularity-adjusted-block-model}}

We will first define the PABM.

\begin{definition}[Popularity Adjusted Block Model]
\label{pabm}
Let $P \in [0, 1]^{n \times n}$ be a symmetric edge probability matrix for a 
set of $n$ 
vertices, $V$. Each vertex has a community label $1, ..., K$, and the rows and 
columns of $P$ are arranged by community label such that $n_k \times n_l$ block 
$P^{(kl)}$ describes the edge probabilities between vertices in communities 
$k$ and $l$ ($P^{(lk)} = (P^{(kl)})^\top$). 
Let graph $G = (V, E)$ be an undirected, unweighted graph such 
that its corresponding adjacency matrix $A \in \{0, 1\}^{n \times n}$ is a 
realization of $\text{Bernoulli}(P)$, i.e., 
$A_{ij} \stackrel{\text{indep}}{\sim}\text{Bernoulli}(P_{ij})$ for $i > j$ 
($A_{ij} = A_{ji}$ and $A_{ii} = 0$). 

If each block $P^{(kl)}$ can be written as the outer product of two vectors:

\begin{equation} \label{eq:pabm}
  P^{(kl)} = \lambda^{(kl)} (\lambda^{(lk)})^\top
\end{equation}

for a set of $K^2$ fixed vectors $\{\lambda^{(st)}\}_{s, t = 1}^K$ where each 
$\lambda^{(st)}$ is a column vector 
of dimension $n_s$, then graph $G$ and its corresponding adjacency matrix $A$ 
is a realization of a popularity adjusted block model with parameters 
$\{\lambda^{(st)}\}_{s, t = 1}^K$. 
\end{definition}

We will use the notation \(A \sim \text{PABM}(\{\lambda^{(kl)}\}_K)\) to
denote a random adjacency matrix \(A\) drawn from a PABM with parameters
\(\lambda^{(kl)}\) consisting of \(K\) underlying communities.

It is trivial to show that the PABM, as well as all graphs such that the
adjacency matrix is drawn such that \(A_{ij} \sim Bernoulli(P_{ij})\),
is a special case of the GRDPG. It can also be shown that the latent
positions of the PABM under the GRDPG framework consists of \(K\)
\(K\)-dimensional subspaces in \(\mathbb{R}^{K^2}\). While there is no
unique latent configuration \(X\) such that \(X X^\top = P\), the edge
probability \(P\) for the PABM, they all have this subspace structure,
and one in particular consists of \emph{orthogonal} subspaces.

\begin{theorem}[Connecting the PABM to the GRDPG for $K = 2$]
\label{theorem1}  
Let 

$$X = \begin{bmatrix} 
\lambda^{(11)} & \lambda^{(12)} & 0 & 0 \\ 
0 & 0 & \lambda^{(21)} & \lambda^{(22)} 
\end{bmatrix}$$

$$U = \begin{bmatrix} 1 & 0 & 0 & 0 \\
0 & 0 & 1 / \sqrt{2} & 1 / \sqrt{2} \\
0 & 0 & 1 / \sqrt{2} & - 1 / \sqrt{2} \\
0 & 1 & 0 & 0 \end{bmatrix}$$

where each $\lambda^{(kl)}$ is a vector as in Definition 1. 
Then $A \sim GRDPG_{3, 1}(X U)$ and $A \sim PABM(\{(\lambda^{(kl)}\}_2)$ are 
equivalent.
\end{theorem}

\begin{theorem}[Generalization to $K > 2$] 
\label{theorem2}
There exists a block diagonal matrix 
$X \in \mathbb{R}^{n \times K^2}$ defined by PABM parameters 
$\{\lambda^{(kl)}\}_K$ and orthonormal matrix 
$U \in \mathbb{R}^{K^2 \times K^2}$ that is fixed 
for each $K$ such that $A \sim GRDPG_{K (K+1) / 2, K (K-1) / 2}(XU)$ and 
$A \sim PABM(\{(\lambda^{(kl)}\})_K)$ are equivalent.
\end{theorem}

\begin{proof}
Define the following matrices from $\{\lambda^{(kl)}\}_K$: 

$$\Lambda^{(k)} = 
\begin{bmatrix} \lambda^{(k,1)} & \cdots & \lambda^{(k, K)} \end{bmatrix}
\in \mathbb{R}^{n_k \times K}$$

\begin{equation} \label{eq:xy}
X = \text{blockdiag}(\Lambda^{(1)}, ..., \Lambda^{(K)}) \in \mathbb{R}^{n \times K^2}
\end{equation}

$$L^{(k)} = \text{blockdiag}(\lambda^{(1k)}, ..., \lambda^{(Kk)}) \in 
\mathbb{R}^{n \times K}$$

$$Y = \begin{bmatrix} L^{(1)} & \cdots & L^{(K)} \end{bmatrix} \in 
\mathbb{R}^{n \times K^2}$$

Then $P = X Y^\top$.

Similar to the $K = 2$ case, we have $Y = X \Pi$ for a permutation matrix
$\Pi$, resulting in $P = X \Pi X^\top$.  
The permutation described by $\Pi$ has $K$ fixed points, which correspond to 
$K$ eigenvalues equal to $1$ with corresponding eigenvectors $e_k$ where 
$k = r (K + 1) + 1$ for $r = 0, ..., K - 1$. It also has 
$\binom{K}{2} = K (K - 1) / 2$ cycles of order $2$. Each cycle corresponds to 
a pair of eigenvalues $+1$ and $-1$ and a pair of eigenvectors 
$(e_s + e_t) / \sqrt{2}$ and $(e_s - e_t) / \sqrt{2}$.

Then $\Pi$ has $K (K + 1) / 2$ eigenvalues equal to $1$ and $K (K - 1) / 2$ 
eigenvalues equal to $-1$. $\Pi$ has the decomposed form 

\begin{equation} \label{eq:permutation}
\Pi = U I_{K (K + 1) / 2, K (K - 1) / 2} U^\top
\end{equation}

The edge probability matrix then can be written as:

\begin{equation} \label{eq:pabm-grdpg}
P = X U I_{p, q} (X U)^\top
\end{equation}

\begin{equation} \label{eq:p}
p = K (K + 1) / 2
\end{equation}

\begin{equation} \label{eq:q}
q = K (K - 1) / 2
\end{equation}

and we can describe the PABM with $K$ communities as a GRDPG with latent 
positions $X U$ with signature $\big( K (K + 1) / 2, K (K - 1) / 2 \big)$.
\end{proof}

\begin{example}[$K = 3$] Using the same notation as in Theorem \ref{theorem2}:

$$X = \begin{bmatrix} 
\lambda^{(11)} & \lambda^{(12)} & \lambda^{(13)} & 0 & 0 & 0 & 0 & 0 & 0 \\
0 & 0 & 0 & \lambda^{(21)} & \lambda^{(22)} & \lambda^{(23)} & 0 & 0 & 0 \\
0 & 0 & 0 & 0 & 0 & 0 & \lambda^{(31)} & \lambda^{(32)} & \lambda^{(33)}
\end{bmatrix}$$

$$Y = \begin{bmatrix} 
\lambda^{(11)} & 0 & 0 & \lambda^{(12)} & 0 & 0 & \lambda^{(13)} & 0 & 0 \\
0 & \lambda^{(21)} & 0 & 0 & \lambda^{(22)} & 0 & 0 & \lambda^{(23)} & 0 \\
0 & 0 & \lambda^{(31)} & 0 & 0 & \lambda^{(32)} & 0 & 0 & \lambda^{(33)}
\end{bmatrix}$$

Then $P = X Y^\top$ and $Y = X \Pi$ where $\Pi$ is a permutation matrix 
consisting of $3$ fixed points and $3$ cycles of order 2:

$$\Pi = \begin{bmatrix} 
1 & 0 & 0 & 0 & 0 & 0 & 0 & 0 & 0 \\
0 & 0 & 0 & 1 & 0 & 0 & 0 & 0 & 0 \\
0 & 0 & 0 & 0 & 0 & 0 & 1 & 0 & 0 \\
0 & 1 & 0 & 0 & 0 & 0 & 0 & 0 & 0 \\
0 & 0 & 0 & 0 & 1 & 0 & 0 & 0 & 0 \\
0 & 0 & 0 & 0 & 0 & 0 & 0 & 1 & 0 \\
0 & 0 & 1 & 0 & 0 & 0 & 0 & 0 & 0 \\
0 & 0 & 0 & 0 & 0 & 1 & 0 & 0 & 0 \\
0 & 0 & 0 & 0 & 0 & 0 & 0 & 0 & 1
\end{bmatrix}$$

* Positions 1, 5, 9 are fixed.

* The cycles of order 2 are $(2, 4)$, $(3, 7)$, and $(6, 8)$.
    
Therefore, we can decompose $\Pi = U I_{6, 3} U^\top$ where the first three 
columns of $U$ consist of $e_1$, $e_5$, and $e_9$ corresponding to the fixed 
positions $1$, $5$, and $9$, the next three columns consist of eigenvectors 
$(e_k + e_l) / \sqrt{2}$, and the last three columns consist of eigenvectors 
$(e_k - e_l) / \sqrt{2}$, where pairs $(k, l)$ correspond to the cycles of 
order 2 described above.

The latent positions are the rows of  
$$XU = \begin{bmatrix}
  \lambda^{(11)} & 0 & 0 & 
  \lambda^{(12)} / \sqrt{2} & \lambda^{(13)} / \sqrt{2} & 0 & 
  \lambda^{(12)} / \sqrt{2} & \lambda^{(13)} / \sqrt{2} & 0 \\
  0 & \lambda^{(22)} & 0 & 
  \lambda^{(21)} / \sqrt{2} & 0 & \lambda^{(23)} / \sqrt{2} & 
  -\lambda^{(21)} / \sqrt{2} & 0 & \lambda^{(23)} / \sqrt{2} \\
  0 & 0 & \lambda^{(33)} & 
  0 & \lambda^{(31)} / \sqrt{2} & \lambda^{(32)} / \sqrt{2} & 
  0 & -\lambda^{(31)} / \sqrt{2} & -\lambda^{(32)} / \sqrt{2}
\end{bmatrix}$$
\end{example}

This leads to the following theorem.

\begin{theorem}
\label{osc}
Let $P = V \Lambda V^\top$ be the spectral decomposition of the edge probability matrix of a PABM. Define $B = n V V^\top$. Then $B_{ij} = 0$ $\forall i, j$ in different communities. 
\end{theorem}

If \(\hat{V}\) is the \emph{unscaled} ASE of \(A\), Theorem \ref{osc}
and results from \citeauthor{rubindelanchy2017statistical} together
imply \(n \hat{V} \hat{V}^\top \stackrel{a.s.}{\to} 0\), leading to the
following result:

\begin{algorithm}[t]
  \DontPrintSemicolon
  \SetAlgoLined
  \KwData{Adjacency matrix $A$, number of communities $K$}
  \KwResult{Community assignments $1, ..., K$}
    Compute the eigenvectors of $A$ that correspond to the $K (K+1) / 2$ most 
    positive eigenvalues and $K (K-1) / 2$ most negative eigenvalues. Construct 
    $V$ using these eigenvectors as its columns.\;
    Compute $B = |n V V^\top|$, applying $|\cdot|$ entry-wise.\;
    Construct graph $G$ using $B$ as its similarity matrix.\;
    Partition $G$ into $K$ disconnected subgraphs  
    (e.g., using edge thresholding or spectral clustering).\;
    Map each partition to the community labels $1, ..., K$.\;
  \caption{Orthogonal Spectral Clustering.}
\end{algorithm}

\begin{theorem}
\label{theorem4} 
Let $\hat{B}_n$ with entries $\hat{B}_n^{(ij)}$ be the affinity matrix from OSC 
(Alg. 1). Then $\forall$ pairs $(i, j)$ belonging to different communities 
and sparsity factor satisfying $n \rho_n = \omega\{(\log n)^{4c}\}$, 

\begin{equation} \label{eq:thm4}
\max_{i, j} |n (\hat{v}_n^{(i)})^\top \hat{v}_n^{(j)}| = 
O_P \Big( \frac{(\log n)^c}{\sqrt{n \rho_n}} \Big)
\end{equation}

This provides the result that $\forall i, j$ in different communities, 
$\hat{B}_n^{(ij)} \stackrel{a.s.}{\to} 0$.
\end{theorem}

Since every ASE of the PABM consists of subspaces and as
\(n \to \infty\) each point of the ASE approaches its subspace almost
surely, SSC should also work for PABM community detection. Combining
this with the results by \citeauthor{jmlr-v28-wang13}, which state that
if the points lie sufficiently close to their respective subspaces and
the cosine of the angles between subspaces is sufficiently small, SDP
will hold. We show that the unscaled ASE exhibits exactly these
conditions for sufficiently large \(n\).

\begin{theorem}
\label{theorem5}
Let $P_n$ describe the edge probability matrix of the PABM with 
$n$ vertices, and let $A_n \sim \text{Bernoulli}(P_n)$.  Let $\hat{V}_n$ be the 
matrix of eigenvectors of $A_n$ corresponding to the $K (K + 1) / 2$ most
positive and $K (K - 1) / 2$ most negative eigenvalues. Then 
$\exists \lambda > 0$ and $N \in \mathbb{N}$ such that when $n > N$, 
$\sqrt{n} \hat{V}_n$ obeys the subspace detection property with probability 1.
\end{theorem}

\hypertarget{manifold-clustering}{%
\subsection{Manifold Clustering}\label{manifold-clustering}}

We would like to extend subspace clustering to manifold clustering. The
problem setup is as follows: Suppose that in the latent space
\(\mathcal{X} \subset \mathbb{R}^d\), sample \(X\) of \(n\) points lie
on a union of \(K\) disjoint manifolds with each manifold corresponding
to a community. If \(A \sim \text{RDPG}(X)\), we wish to recover the
community labels (up to permutation) from \(A\).

Similarly, suppose that probability distribution \(F\) is described as
follows:

\begin{enumerate}
\def\labelenumi{\arabic{enumi}.}
\tightlist
\item
  Define functions \(f_1, ..., f_K\) such that
  \(f_k : [0, 1] \mapsto \mathcal{X}\) and \(f_i(t) \neq f_j(t)\)
  \(\forall i \in [K]\) and \(x \in [0, 1]\).
\item
  Sample labels
  \(z_1, ..., z_n \stackrel{iid}{\sim} Categorical(\pi_1, ..., \pi_K)\).
\item
  Sample \(t_1, ..., t_n \stackrel{iid}{\sim} D\) where \(D\) has
  support \([0, 1]\).
\item
  Set latent positions \(x_i = f_{z_i}(t_i)\) and
  \(X = \begin{bmatrix} x_1 & \cdots & x_n \end{bmatrix}^\top\).
\end{enumerate}

Then if \((A, X) \sim \text{RDPG}(F, n)\) and we observe \(A\), we wish
to recover hidden labels \(z_1, ..., z_n\).

\hypertarget{affine-subspaces}{%
\subsubsection{Affine Subspaces}\label{affine-subspaces}}

We will motivate an approach by the following example.

\begin{example}
Let $U_1, ..., U_n \stackrel{iid}{\sim} Uniform(0, 1)$ with order statistics 
$U_{(1)}, ..., U_{(n)}$. Then $\forall a \in (0, 1)$ and $\delta \in (0, 1)$, $\exists N = N(\delta, a) < \infty$ such that $\forall n \geq N$, 

\begin{equation}
P(\max_i U_{(i+1)} - U_{(i)} \leq a) \geq 1 - \delta / 2
\end{equation}

Where $N(\delta, a)$ is monotone increasing w.r.t. $\delta$ and $a$. To prove this, we start with the fact that $U_{(i+1)} - U_{(i)} \sim Beta(1, n)$. Then 

\begin{equation}
P(U_{(i+1)} - U_{(i)} \leq a) = 1 - (1 - a)^n
\end{equation}

and 

\begin{equation}
\label{eq:unsolvable}
P(\max_i U_{(i+1)} - U_{(i)} \leq a) \geq (P(U_{(i+1)} - U_{(i)} \leq a))^n = (1 - (1 - a)^n)^n
\end{equation}

This expression is monotone increasing $\forall n \geq N_1$ for some $N_1 < \infty$. 
Setting $(1 - (1 - a)^{N_2})^{N_2} \geq 1 - \delta / 2$, we can solve for a finite ${N_2}$. Then $N = \max(N_1, N_2)$.
\end{example}

If we extend this example such that \(n_1\) points are sampled uniformly
from the segment \(f_1(t) = (t, 0)\) and \(n_2\) points are sampled
uniformly from the segment \(f_2(t) = (t, a)\) for \(t \in [0, 1]\),
then a sample of size \(N(\delta, a)\) is sufficient to satisfy:

\begin{equation}
\begin{split}
P(\max_i X_{(i+1)} - X_{(i)} \leq \min_{i, j} \|X_i - Y_j\|) \geq 1 - \delta \\
P(\max_j Y_{(j+1)} - Y_{(j)} \leq \min_{i, j} \|X_i - Y_j\|) \geq 1 - \delta
\end{split}
\end{equation}

for \(X_i\) in the first segment and \(Y_j\) in the second segment and
\(X_{(i)}\), \(Y_{(j)}\) are order statistics in the first coordinate.
If each segment corresponds to a community, this leads to the following
two results:

\begin{enumerate}
\def\labelenumi{\arabic{enumi}.}
\item
  Single linkage clustering with \(K = 2\) will perform perfect
  community detection with probability at least \(1 - \delta\).
\item
  An \(\epsilon\)-neighborhood graph with \(\epsilon \in (0, a)\) will
  consists of at least 2 disjoint subgraphs such that no subgraph
  consists of members of two different communities (analogous to the
  SDP), with probability at least \(1 - \delta\).
\end{enumerate}

We can then further extend this to the case where points are drawn from
unit segments with noise.

\hypertarget{mainfold-learning}{%
\subsubsection{Mainfold Learning}\label{mainfold-learning}}

If instead of sampling uniformly from line segments of unit length, we
sample uniformly from a 1 dimensional manifolds of unit length, the
above property still holds. Let
\(U_1, ..., U_n \stackrel{iid}{\sim} Uniform(0, 1)\) and
\(f : [0, 1] \mapsto \mathbb{R}^d\) be a smooth function such that
\(\int_0^1 \sqrt{1 - f'(t)} dt = 1\) and
\(\int_u^v \sqrt{1 - f'(t)} dt = \|u - v\|\). Then
\(U_{(i+1)} - U_{(i)} > \|f(U_{(i+1)}) - f(U_{(i)})\|\), so
\(P(U_{(i+1)} - U_{(i)} \leq \alpha) \leq P(\|f(U_{(i+1)}) - f(U_{(i)})\| \leq a)\).
If the shortest distance between the two manifolds is \(a\), then the
same \(N\) as before is sufficient, although perhaps a more lenient
lower bound can be derived based on the shape of \(f_i(\cdot)\).

\hypertarget{summary}{%
\section{Summary}\label{summary}}

\hypertarget{estimated-timeline-of-completion}{%
\section{Estimated Timeline of
Completion}\label{estimated-timeline-of-completion}}

Literature review: August 2021

Complete proofs of main theorems: January 2022

Simulations and real data analyses: March 2022

Dissertation completion: April 2022

\newpage

\renewcommand\refname{References}
  \bibliography{proposal.bib}

\end{document}
